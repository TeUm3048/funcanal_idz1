\documentclass{amsart}

\usepackage[utf8]{inputenc}
\usepackage[T2A]{fontenc}
\usepackage[english,russian]{babel}
\usepackage{amsthm,amsmath,amsfonts,amssymb}
\usepackage{fullpage}
\usepackage{eufrak}
\usepackage{bbm}

%%% Дополнительная работа с математикой
\usepackage{amsfonts,amssymb,amsthm,mathtools} % AMS
\usepackage{amsmath}
\usepackage{icomma}

%% Шрифты
\usepackage{euscript}	% Шрифт Евклид
\usepackage{mathrsfs}	% Красивый матшрифт

%% Python
\usepackage{pythontex} 


%% Свои команды
\DeclareMathOperator{\lb}{\mathop{lb}}	% логарифм по основанию 2
\DeclareMathOperator{\sgn}{\mathop{sgn}}	% сигнум
\renewcommand{\Im}{\mathop{\mathrm{Im}}\nolimits}	% мнимая часть
\renewcommand{\Re}{\mathop{\mathrm{Re}}\nolimits}	% вещественная часть
\renewcommand{\emptyset}{\varnothing}	% пустое множество
\renewcommand{\le}{\leqslant}	% отечественная версия "меньше или равно"
\renewcommand{\ge}{\geqslant}	% отечественная версия "больше или равно"
\renewcommand{\epsilon}{\varepsilon}	% стандартная "эпсилон"
\renewcommand{\phi}{\varphi}	% стандартная "фи"
\newcommand{\const}{\mathrm{const}}	% константа
\newcommand{\transpose}{^{\mathsf{T}}} % транспонирование

%% Множества чисел
\DeclareMathOperator{\Natural}{\mathbb{N}}	% Натуральные числа
\DeclareMathOperator{\Integer}{\mathbb{Z}}	% Целые числа
\DeclareMathOperator{\Integerp}{\mathbb{Z}_{+}}	% Целые неотрицательные числа
\DeclareMathOperator{\Rational}{\mathbb{Q}}	% Рациональные числа
\DeclareMathOperator{\Real}{\mathbb{R}}	% Вещественные числа
\DeclareMathOperator{\Realp}{\mathbb{R}_{>0}}	% Вещественные положительные числа
\DeclareMathOperator{\Realn}{\mathbb{R}_{<0}}	% Вещественные отрицательные числа
\DeclareMathOperator{\Realnn}{\mathbb{R}_{\ge 0}}	% Вещественные неотрицательные числа
\DeclareMathOperator{\Realnp}{\mathbb{R}_{\le 0}}	% Вещественные неположительные числа
\DeclareMathOperator{\Complex}{\mathbb{C}}	% Комплексные числа

%% Заглавные греческие буквы
\DeclareMathOperator{\Alpha}{\mathrm{A}}	% Альфа
\DeclareMathOperator{\Beta}{\mathrm{B}}	% Вета
\DeclareMathOperator{\Epsilon}{\mathrm{E}}	% Эпсилон
\DeclareMathOperator{\Zeta}{\mathrm{Z}}	% Дзета
\DeclareMathOperator{\Eta}{\mathrm{H}}	% Эта
\DeclareMathOperator{\Iota}{\mathrm{I}}	% Йота
\DeclareMathOperator{\Kappa}{\mathrm{K}}	% Каппа
\DeclareMathOperator{\Mu}{\mathrm{M}}	% Мю
\DeclareMathOperator{\Nu}{\mathrm{N}}	% Ню
\DeclareMathOperator{\Omicron}{\mathrm{O}}	% Омикрон
\DeclareMathOperator{\Rho}{\mathrm{P}}	% Ро
\DeclareMathOperator{\Tau}{\mathrm{T}}	% Тау
\DeclareMathOperator{\Chi}{\mathrm{X}}	% Хи

%% Теория вероятностей
\renewcommand{\Prob}{\mathbb P}	% вероятность
\newcommand{\Expect}{\mathbb E}	% математическое ожидание
\renewcommand{\Variance}{\mathbb D}	% дисперсия
\newcommand{\Entropy}{\mathbb H}	% энтропия
\DeclareMathOperator{\cov}{\mathop{cov}}	% ковариация
\DeclareMathOperator{\supp}{\mathop{supp}}	% носитель
\DeclareMathOperator{\Skewness}{\mathop{Skew}}	% коэффициент асимметрии
\DeclareMathOperator{\Kurtosis}{\mathop{Kurt}}	% коэффициент эксцесса

%%% Статистический анализ
\newcommand*{\moment}[1]{\overline{#1}}	% выборочный момент
\DeclareMathOperator{\hskew}{\mathop{\widehat{Skew}}}	% выборочный коэффициент асимметрии
\DeclareMathOperator{\hkurt}{\mathop{\widehat{Kurt}}}	% выборочный коэффициент эксцесса
%% Однопараметрические распределения
\newcommand*{\chisq}[1]{\chi^2_{#1}}	% Распределение хи-квадрат
\newcommand*{\Stud}[1]{\mathcal{S}_{#1}}	% Распределение Стьюдента
\newcommand*{\Exp}[1]{\mathop{\mathrm{Exp}}(#1)}	% Показательное распределение
\newcommand*{\Bern}[1]{\mathop{\mathrm{Bern}}(#1)}	% Распределение Бернулли
\newcommand*{\Geom}[1]{\mathop{\mathrm{Geom}}(#1)}	% Геометрическое распределение
\newcommand*{\Pois}[1]{\mathop{\mathrm{Pois}}(#1)}	% Распределение Пуассона
%% Двухпараметрические распределения
\newcommand*{\FS}[2]{\mathcal{F}_{#1, #2}}	% Распределение Фишера-Снедекора
\newcommand*{\Norm}[2]{\mathcal{N}(#1, #2)}	% Нормальное распределение
\newcommand*{\Unif}[2]{\mathcal{U}(#1, #2)}	% Равномерное распределение
\newcommand*{\DE}[2]{\mathop{\mathrm{DE}}(#1, #2)}	% Распределение Лапласа
\newcommand*{\Cauchy}[2]{\mathop{\mathrm{C}}(#1, #2)}	% Распределение Коши
\newcommand*{\Binom}[2]{\mathop{\mathrm{Binom}}(#1, #2)}	% Биномиальное распределение
\newcommand*{\Betadist}[2]{\mathop{\mathrm{Beta}}(#1, #2)}	% Бета-распределение
\newcommand*{\Gammadist}[2]{\mathop{\mathrm{Gamma}}(#1, #2)}	% Гамма-распределение
%% Ажурные и готические буквы
\newcommand*{\Acl}{\mathcal{A}}	% A красивое
\newcommand*{\Ccl}{\mathcal{C}}	% C красивое
\newcommand*{\Fcl}{\mathcal{F}}	% F красивое
\newcommand*{\Icl}{\mathcal{I}}	% I красивое
\newcommand*{\Kcl}{\mathcal{K}}	% K красивое
\newcommand*{\Pcl}{\mathcal{P}}	% P красивое
\newcommand*{\Ycl}{\mathcal{Y}}	% Y красивое
\newcommand*{\Afr}{\mathfrak{A}}	% A готическое
\newcommand*{\Bfr}{\mathfrak{B}}	% B готическое
\newcommand*{\Ffr}{\mathfrak{F}}	% F готическое
\newcommand*{\Kfr}{\mathfrak{K}}	% K готическое
\newcommand*{\Xfr}{\mathfrak{X}}	% X готическое
%% Теория оценивания
\newcommand*{\ind}[1]{\mathbbm{1}_{\lbrace #1 \rbrace}}	% индикаторная функция
\newcommand*{\bias}[2]{\mathop{\mathrm{bias}}\nolimits_{#1}(#2)}	% смещение

%% Перенос знаков в формулах (по Львовскому)
\newcommand*{\hm}[1]{#1\nobreak\discretionary{}
	{\hbox{$\mathsurround=0pt #1$}}{}}

%%% Работа с картинками
\usepackage{graphicx,xcolor}	% Для вставки рисунков
\graphicspath{{images/}{images2/}}	% папки с картинками
\setlength\fboxsep{3pt}	% Отступ рамки \fbox{} от рисунка
\setlength\fboxrule{1pt}	% Толщина линий рамки \fbox{}
\usepackage{wrapfig}	% Обтекание рисунков и таблиц текстом
\RequirePackage{caption}
\DeclareCaptionLabelSeparator{defffis}{ --- }
\captionsetup{justification=centering,labelsep=defffis}
\usepackage{float}
\usepackage{tikz}
\usepackage{pgfplots}
\pgfplotsset{compat=newest}
\usetikzlibrary{patterns}
\usetikzlibrary{calc}
\usepgfplotslibrary{fillbetween}


%%% Работа с таблицами
\usepackage{array,tabularx,tabulary,booktabs}	% Дополнительная работа с таблицами
\usepackage{longtable}	% Длинные таблицы
\usepackage{multirow}	% Слияние строк в таблице
\usepackage{makecell}
\usepackage{multicol}


\renewcommand{\qedsymbol}{}

\renewcommand*{\arraystretch}{1.25}

\newtheorem{problem}{Задание}

\begin{document}
	\newcommand{\problemset}[1]{
		\begin{center}
			\Large #1
		\end{center}
	}

	\begin{tabbing}
	\hspace{11cm} \= Студент: \= Рыжиков Иван  \\	% не забудьте исправить, студент Вы или студентка :)
																									% (а то некоторые забывают)
	\> Группа: \> 2381 \\	% Здесь меняете № группы
	\> Вариант: \> 17 \\		% А здесь меняете № варианта
	\> Дата: \> \today		% А вот здесь ничего не меняем!!!
\end{tabbing}
\hrule
\vspace{1cm}	% в данном файле меняем только Пол, Фамилию Имя, № группы и № варианта
	\problemset{Элементы функционального анализа}
\problemset{Индивидуальное домашнее задание №1}	% поменяйте номер ИДЗ

\renewcommand*{\proofname}{Решение}

%%%%%%%%%%%%%% ЗАДАНИЕ №1 %%%%%%%%%%%%%%
%% Условие задания №1

\begin{sympycode}
v1 = Matrix([3, 6, 0])
v2 = Matrix([5, 0, 2])
v3 = Matrix([0, 6, 8])
v4 = Matrix([Rational(43, 8), 0, 0])
v5 = Matrix([0, Rational(102, 11), 0])
v6 = Matrix([0, 0, Rational(35, 3)])

# x < 0, y, z ≥ 0
v7 = Matrix([-v1[0], v1[1], v1[2]])
v8 = Matrix([-v2[0], v2[1], v2[2]])
v9 = Matrix([-v4[0], v4[1], v4[2]])

# y < 0, x, z ≥ 0
v10 = Matrix([v1[0], -v1[1], v1[2]])
v11 = Matrix([v3[0], -v3[1], v3[2]])
v12 = Matrix([v5[0], -v5[1], v5[2]])

# z < 0, x, y ≥ 0
v13 = Matrix([v2[0], v2[1], -v2[2]])
v14 = Matrix([v3[0], v3[1], -v3[2]])
v15 = Matrix([v6[0], v6[1], -v6[2]])

# x,y,z <= 0
v16 = Matrix([-v1[0], -v1[1], v1[2]])
v17 = Matrix([-v2[0], v2[1], -v2[2]])
v18 = Matrix([v3[0], -v3[1], -v3[2]])

# vectors
a = Matrix([5, -7, 9])
b = Matrix([-9, 7, 5])

verticals = [
    ("v1", v1), 
    ("v2", v2), 
    ("v3", v3), 
    ("v4", v4), 
    ("v5", v5), 
    ("v6", v6),
    ("v7", v7),
    ("v8", v8),
    ("v9", v9),
    ("v10", v10),
    ("v11", v11),
    ("v12", v12),
    ("v13", v13),
    ("v14", v14),
    ("v15", v15),
    ("v16", v16),
    ("v17", v17),
    ("v18", v18),
]

verticals_without_names = [v[1] for v in verticals]
\end{sympycode}

\begin{problem}
Определить $\left\Vert a \right\Vert$, $\left\Vert b \right\Vert$,
$\left\Vert a + b \right\Vert$ в норме Минковского, порожденную многогранником $W$.

Многогранник $W$ задан вершинами в первом октанте, остальные вершины получаются из них симметричным отражением относительно координатных плоскостей.

\(v_1 = \sympy{v1}\), \(v_2 = \sympy{v2}\), \(v_3 = \sympy{v3}\), \(v_4 = \sympy{v4}\), \(v_5 = \sympy{v5}\), \(v_6 = \sympy{v6}\).

\(a = \sympy{a}\), \(b = \sympy{b}\).
\end{problem}

\begin{proof}\hfill

    \subsection*{Определение нормы}

    Норма — функция $|x|: X \rightarrow \mathbf{R}$, обладающая следующими свойствами:

    \begin{enumerate}
        \item $|x| \geq 0$ и $|x| = 0 \Leftrightarrow x = 0$,
        \item $|\alpha x| = |\alpha| |x|$,
        \item $|x + y| \leq |x| + |y|$.
    \end{enumerate}

    \subsection*{Норма Минковского}

    Пусть задано множество $W$ в линейном пространстве $X$, такое что

    \begin{enumerate}
        \item $W$ — выпуклое множество
        \item $0$ — внутренняя точка и точка симметрии $W$
        \item $\forall x \in X, x \neq 0\ \exists k > 0: \frac{1}{k} x \in W$
    \end{enumerate}

    Тогда норма Минковского определяется как

    \[
        \left\Vert x \right\Vert = \inf \{ k > 0: \frac{x}{k} \in W \}
    \]

    Симметричные точки:
    \[
        \begin{aligned}
             & \sympy{verticals[0:6]}                       \\
             & x < 0, y, z \geq 0: \sympy{verticals[6:9]}   \\
             & y < 0, x, z \geq 0: \sympy{verticals[9:12]}  \\
             & z < 0, x, y \geq 0: \sympy{verticals[12:15]} \\
             & x, y, z \leq 0: \sympy{verticals[15:18]}     \\
        \end{aligned}
    \]
    \begin{sympycode}
plane_verticals_in_octant_1 = [[v1, v2, v3], [v1, v2, v4], [v1, v3, v5], [v2, v3, v6]] # +++
plane_verticals_in_octant_2 = [[Matrix([v[0], -v[1], v[2]]) for v in plane] for plane in plane_verticals_in_octant_1] # +-+
plane_verticals_in_octant_3 = [[Matrix([v[0], -v[1], -v[2]]) for v in plane] for plane in plane_verticals_in_octant_1] # +--
plane_verticals_in_octant_4 = [[Matrix([v[0], v[1], -v[2]]) for v in plane] for plane in plane_verticals_in_octant_1] # ++-
plane_verticals_in_octant_5 = [[Matrix([-v[0], v[1], v[2]]) for v in plane] for plane in plane_verticals_in_octant_1] # -++
plane_verticals_in_octant_6 = [[Matrix([-v[0], -v[1], v[2]]) for v in plane] for plane in plane_verticals_in_octant_1] # --+
plane_verticals_in_octant_7 = [[Matrix([-v[0], -v[1], -v[2]]) for v in plane] for plane in plane_verticals_in_octant_1] # ---
plane_verticals_in_octant_8 = [[Matrix([-v[0], v[1], -v[2]]) for v in plane] for plane in plane_verticals_in_octant_1] # -+-

plane_verticals = [*plane_verticals_in_octant_1, *plane_verticals_in_octant_2, *plane_verticals_in_octant_3, *plane_verticals_in_octant_4, *plane_verticals_in_octant_5, *plane_verticals_in_octant_6, *plane_verticals_in_octant_7, *plane_verticals_in_octant_8]
\end{sympycode}

\vfill

    Разобьем плоскости на октанты:

    Первый октант: \\
    \(\sympy{plane_verticals_in_octant_1}\) \\

    Второй октант: \\
    \(\sympy{plane_verticals_in_octant_2}\) \\

    Третий октант: \\
    \(\sympy{plane_verticals_in_octant_3}\) \\

    Четвертый октант: \\
    \(\sympy{plane_verticals_in_octant_4}\) \\

    Пятый октант: \\
    \(\sympy{plane_verticals_in_octant_5}\) \\

    Шестой октант: \\
    \(\sympy{plane_verticals_in_octant_6}\) \\

    Седьмой октант: \\
    \(\sympy{plane_verticals_in_octant_7}\) \\

    Восьмой октант: \\
    \(\sympy{plane_verticals_in_octant_8}\)

    \begin{sympycode}
def plane_equation_3points_sympy(p1, p2, p3):
    """
    По трем точкам p1, p2, p3 строит уравнение плоскости A*x + B*y + C*z + D = 0,
    где D < 0, используя Sympy.
    
    Параметры:
    p1, p2, p3: кортежи или списки из 3 чисел, например (x, y, z).
    
    Возвращает:
    (A, B, C, D): коэффициенты уравнения плоскости.
    """
    P1 = Matrix(p1)
    P2 = Matrix(p2)
    P3 = Matrix(p3)
    
    v1 = P2 - P1
    v2 = P3 - P1
    
    normal = v1.cross(v2)  
    A, B, C = normal
    
    # Находим D, подставляя любую из точек (например, P1) в уравнение плоскости
    # A*x + B*y + C*z + D = 0  =>  D = - (A*x1 + B*y1 + C*z1)
    D = -(A * P1[0] + B * P1[1] + C * P1[2])
    
    if D > 0:
        A, B, C, D = -A, -B, -C, -D

    return (A, B, C, D)
\end{sympycode}

    \begin{sympycode}
plane_in_octant1 = [plane_equation_3points_sympy(*plane) for plane in plane_verticals_in_octant_1]
plane_in_octant2 = [plane_equation_3points_sympy(*plane) for plane in plane_verticals_in_octant_2]
plane_in_octant3 = [plane_equation_3points_sympy(*plane) for plane in plane_verticals_in_octant_3]
plane_in_octant4 = [plane_equation_3points_sympy(*plane) for plane in plane_verticals_in_octant_4]
plane_in_octant5 = [plane_equation_3points_sympy(*plane) for plane in plane_verticals_in_octant_5]
plane_in_octant6 = [plane_equation_3points_sympy(*plane) for plane in plane_verticals_in_octant_6]
plane_in_octant7 = [plane_equation_3points_sympy(*plane) for plane in plane_verticals_in_octant_7]
plane_in_octant8 = [plane_equation_3points_sympy(*plane) for plane in plane_verticals_in_octant_8]

planes = [*plane_in_octant1, *plane_in_octant2, *plane_in_octant3, *plane_in_octant4, *plane_in_octant5, *plane_in_octant6, *plane_in_octant7, *plane_in_octant8]
\end{sympycode}



    \begin{sympycode}
def print_plane_table(planes):
    num_cols = 1 + len(planes[0])

    print(r'\begin{tabular}{|c|' + 'c|' * (num_cols - 1) + '}')
    print(r'\hline')

    # Header row: blank cell on left, then each point from right to left
    header_cells = [r'\textbf{№}', r'A', r'B', r'C', r'D']
    print(' & '.join(header_cells) + r'\\')
    print(r'\hline')

    # Now each plane gets a row
    for (i, coeffs) in enumerate(planes):
        coeffs_str = [str(c) for c in coeffs]

        row_str = rf'{i + 1}' + ' & ' + ' & '.join(coeffs_str) + r'\\'
        print(row_str)
        print(r'\hline')

    print(r'\end{tabular}')    
\end{sympycode}

\vfill

    \subsection*{Плоскости}

    Найдём коэффициенты нормальное уравнения плоскости для каждой плоскости в октантах:

    \(A \cdot x + B \cdot y + C \cdot z + D = 0\).

    Выберем $D \leq 0$, так чтобы при подстановке любой точки из многогранника $W$ в уравнение плоскости, левая часть была отрицательной.\\

    \begin{sympycode}
print_plane_table(planes)
\end{sympycode}
\vfill

    \subsection*{Проверка выпуклости} \hfill

    Проверим выпуклость многогранника $W$, используя уравнения плоскостей.

    Так как мы выбрали $D \leq 0$, то если все точки многогранника $W$ при подстановке в уравнения плоскостей будут давать отрицательные значения, то многогранник $W$ выпуклый.

    \begin{sympycode}
def print_latex_table(planes, points):
    # Build the LaTeX table. We'll print out a tabular environment.
    # The number of columns = 1 (for the "Plane" column) + len(points).
    num_cols = 1 + len(points)

    print(r'\begin{tabular}{|c|' + 'c|' * len(points) + '}')
    print(r'\hline')

    # Header row: blank cell on left, then each point from right to left
    header_cells = [r'\textbf{Плоскость\textbackslash Вершины}'] + [
        f"${name}$" for (name, (px,py,pz)) in points
    ]
    print(' & '.join(header_cells) + r'\\')
    print(r'\hline')

    # Now each plane gets a row
    for (i, (A,B,C,D)) in enumerate(planes):
        # Create a symbolic or purely numeric representation of plane
        # e.g. 48x + 22y - 18z - 276 = 0
        plane_str = rf'${i + 1}$'
        # Evaluate the left-hand side A*x + B*y + C*z + D at each point
        vals = []
        for (name, (px,py,pz)) in points:
            val = A*px + B*py + C*pz + D
            vals.append(str(val))
        # Build the row: plane equation, then each substitution result
        row_str = plane_str + ' & ' + ' & '.join(vals) + r'\\'
        print(row_str)
        print(r'\hline')

    print(r'\end{tabular}')    
\end{sympycode}
    {\tiny
    \begin{sympycode}
print_latex_table(planes, verticals[0:6])
print("\n")
print_latex_table(planes, verticals[6:12])
print("\n")
print_latex_table(planes, verticals[12:18])
    \end{sympycode}
    }
    \normalsize

    \begin{sympycode}
def calc_norm(v, plane_points, verbose=False):
    """
    v: A Sympy Matrix or vector to test (e.g., Matrix([1, 2, 3])) 
    plane_points: A 3-tuple (u1, u2, u3) of Sympy vectors 
                 describing your plane, such that:
                    u1, u2, u3 = plane_points
    verbose: If True, prints LaTeX logs of each step.
    """
    # Unpack plane points
    u1, u2, u3 = plane_points

    # Step 1: Cross products
    w1 = u2.cross(u3)
    w2 = u1.cross(u3)
    w3 = u1.cross(u2)

    if verbose:
        print(r"Точки плоскости: \(u_1 = " + latex(u1) + r"\), \(u_2 = " + latex(u2) + r"\), \(u_3 = " + latex(u3) + r"\)")
        print()

    # Step 2: Divide each w_i by its dot product with the corresponding u_i
    w1 = w1 / (w1.dot(u1))
    w2 = w2 / (w2.dot(u2))
    w3 = w3 / (w3.dot(u3))

    if verbose:
        print(r"Вычисление биортогонального базиса:")
        print(r"\[w_{1} = \frac{u_{2} \times u_{3}}{w_{1}\cdot u_{1}} = " + latex(w1) + r"\]")
        print(r"\[w_{2} = \frac{u_{1} \times u_{3}}{w_{2}\cdot u_{2}} = " + latex(w2) + r"\]")
        print(r"\[w_{3} = \frac{u_{1} \times u_{2}}{w_{3}\cdot u_{3}} = " + latex(w3) + r"\]")
        print()

    # Step 3: Build transformation matrix and transform v
    W = Matrix([list(w1), list(w2), list(w3)])
    x_W = W * v

    if verbose:
        print(r"Построение матрицы \(W\) и преобразование вектора \(v\):")
        print(r"\[W = \begin{pmatrix}")
        for row in W.tolist():
            print(" & ".join(str(elem) for elem in row) + r"\\")
        print(r"\end{pmatrix}\quad,\quad v = " + latex(v) + r"\]")
        print(r"\[x_{W} = W\cdot v = " + latex(x_W) + r"\]")
        print()

    # Step 4: Check for nonnegative coordinates
    if x_W[0] >= 0 and x_W[1] >= 0 and x_W[2] >= 0:
        result = x_W[0] + x_W[1] + x_W[2]
        if verbose:
            print(r"\textbf{Все координаты неотрицательны; норма вектора:}")
            print(r"\[ ||v|| = x_{W,1} + x_{W,2} + x_{W,3} = " + latex(result) + r"\]")
            print()
        return result
    else:
        if verbose:
            print(r"\textbf{Есть отрицательные значения. Не этот конус.}")
            print()
        return None

def calc_norm_W(v, planes_as_points, verbose=False):
    """
    v: A Sympy vector to test
    planes_as_points: A list of plane descriptions, each a 3-tuple (u1, u2, u3)
    verbose: If True, prints logs of the process.
    """
    if verbose:
        print(r"Поиск нормы вектора \(v = " + latex(v) + r"\):")
        print('\n')

    for i, plane in enumerate(planes_as_points):
        if verbose:
            print(r"\textbf{Номер плоскости " + str(i + 1) + "}")
        norm_value = calc_norm(v, plane, verbose=verbose)
        if norm_value is not None:
            if verbose:
                print(r"\( \left\Vert" + latex(v) + r"\right\Vert = " + latex(norm_value) + r"\)\par\medskip")
            return i, norm_value

    if verbose:
        print(r"\textbf{No plane gave a nonnegative combination. Returning (None, None).}")
    return None, None
\end{sympycode}

    \subsection*{Определение норм векторов}

    Проверим, лежит ли заданный вектор в конической оболочке,
    которую образуют три вектора, соответствующие грани многогранника, а именно: выражается ли он как их неотрицательная линейная комбинация

    Для этого постоем биортагональный базисов, так что при умножении на матрицу сразу получаются
    те самые коэффициенты \(\lambda_i\). Если все \(\lambda_i\) неотрицательны, 
    то вектор действительно принадлежит конусу.

    Затем сумма \(\lambda_1 + \lambda_2 + \lambda_3\) интерпретируется как «коэффициент масштабирования», 
    который в задаче равен норме Минковского для данного вектора.

    Найдем норму вектора \(a = \sympy{a}\).

    % Now we can call these functions in a PythonTeX block to see the LaTeX logs:
    \begin{sympycode}
_, val_a = calc_norm_W(a, plane_verticals, verbose=True)
_, val_b = calc_norm_W(b, plane_verticals, verbose=False)
_, val_ab = calc_norm_W(a + b, plane_verticals, verbose=False)
\end{sympycode}

    Аналогично, для векторов \(b = \sympy{b}\) и \(a + b = \sympy{a + b}\).

    Ответ: \[||a|| = \left\Vert\sympy{a} \right\Vert = \sympy{val_a.evalf(5)}\],
    \[||b|| = \left\Vert\sympy{b} \right\Vert = \sympy{val_b.evalf(5)}\],
    \[||a+b|| = \left\Vert\sympy{a + b} \right\Vert = \sympy{val_ab.evalf(5)}\].
\end{proof}	% для удобства создаём по аналогии файлы ihw1.tex, ihw2.tex, etc
										% и просто меняем имя при компиляции
\end{document}